%\documentclass[landscape,a0b,final,a4resizeable]{a0poster}
\documentclass[landscape,a0b,final]{a0poster}
%\documentclass[portrait,a0b,final,a4resizeable]{a0poster}
%\documentclass[portrait,a0b,final]{a0poster}
%%% Option "a4resizeable" makes it possible ot resize the
%   poster by the command: psresize -pa4 poster.ps poster-a4.ps
%   For final printing, please remove option "a4resizeable" !!

\usepackage{epsfig}
\usepackage{multicol}
\usepackage[font=footnotesize]{subfig}
\usepackage{float}
\usepackage{setspace}
\usepackage{amsmath, amssymb, bm}
\DeclareMathOperator*{\argmin}{arg\,min}
\onehalfspacing
\usepackage{mathtools,amsmath,amssymb}
\usepackage{pstricks,pst-grad}



%%%%%%%%%%%%%%%%%%%%%%%%%%%%%%%%%%%%%%%%%%%
% Definition of some variables and colors
%\renewcommand{\rho}{\varrho}
%\renewcommand{\phi}{\varphi}
\setlength{\columnsep}{3cm}
\setlength{\columnseprule}{2mm}
\setlength{\parindent}{0.0cm}



%%%%%%%%%%%%%%%%%%%%%%%%%%%%%%%%%%%%%%%%%%%%%%%%%%%%
%%%               Background                     %%%
%%%%%%%%%%%%%%%%%%%%%%%%%%%%%%%%%%%%%%%%%%%%%%%%%%%%

\newcommand{\background}[3]{
  \newrgbcolor{cgradbegin}{#1}
  \newrgbcolor{cgradend}{#2}
  \psframe[fillstyle=gradient,gradend=cgradend,
  gradbegin=cgradbegin,gradmidpoint=#3](0.,0.)(1.\textwidth,-1.\textheight)
}



%%%%%%%%%%%%%%%%%%%%%%%%%%%%%%%%%%%%%%%%%%%%%%%%%%%%
%%%                Poster                        %%%
%%%%%%%%%%%%%%%%%%%%%%%%%%%%%%%%%%%%%%%%%%%%%%%%%%%%

\newenvironment{poster}{
  \begin{center}
  \begin{minipage}[c]{0.98\textwidth}
}{
  \end{minipage} 
  \end{center}
}



%%%%%%%%%%%%%%%%%%%%%%%%%%%%%%%%%%%%%%%%%%%%%%%%%%%%
%%%                pcolumn                       %%%
%%%%%%%%%%%%%%%%%%%%%%%%%%%%%%%%%%%%%%%%%%%%%%%%%%%%

\newenvironment{pcolumn}[1]{
  \begin{minipage}{#1\textwidth}
  \begin{center}
}{
  \end{center}
  \end{minipage}
}



%%%%%%%%%%%%%%%%%%%%%%%%%%%%%%%%%%%%%%%%%%%%%%%%%%%%
%%%                pbox                          %%%
%%%%%%%%%%%%%%%%%%%%%%%%%%%%%%%%%%%%%%%%%%%%%%%%%%%%

\newrgbcolor{lcolor}{0. 0. 0.80}
\newrgbcolor{gcolor1}{1. 1. 1.}
\newrgbcolor{gcolor2}{.80 .80 1.}

\newcommand{\pbox}[4]{
\psshadowbox[#3]{
\begin{minipage}[t][#2][t]{#1}
#4
\end{minipage}
}}



%%%%%%%%%%%%%%%%%%%%%%%%%%%%%%%%%%%%%%%%%%%%%%%%%%%%
%%%                myfig                         %%%
%%%%%%%%%%%%%%%%%%%%%%%%%%%%%%%%%%%%%%%%%%%%%%%%%%%%
% \myfig - replacement for \figure
% necessary, since in multicol-environment
% \figure won't work

\newcommand{\myfig}[3][0]{
\begin{center}
  \vspace{1.5cm}
  \includegraphics[width=#3\hsize,angle=#1]{#2}
  \nobreak\medskip
\end{center}}



%%%%%%%%%%%%%%%%%%%%%%%%%%%%%%%%%%%%%%%%%%%%%%%%%%%%
%%%                mycaption                     %%%
%%%%%%%%%%%%%%%%%%%%%%%%%%%%%%%%%%%%%%%%%%%%%%%%%%%%
% \mycaption - replacement for \caption
% necessary, since in multicol-environment \figure and
% therefore \caption won't work

%\newcounter{figure}
\setcounter{figure}{1}
\newcommand{\mycaption}[1]{
  \vspace{0.5cm}
  \begin{quote}
    {{\sc Figure} \arabic{figure}: #1}
  \end{quote}
  \vspace{1cm}
  \stepcounter{figure}
}



%%%%%%%%%%%%%%%%%%%%%%%%%%%%%%%%%%%%%%%%%%%%%%%%%%%%%%%%%%%%%%%%%%%%%%
%%% Begin of Document
%%%%%%%%%%%%%%%%%%%%%%%%%%%%%%%%%%%%%%%%%%%%%%%%%%%%%%%%%%%%%%%%%%%%%%

\begin{document}

\background{1. 1. 1.}{1. 1. 1.}{0.5}

\vspace*{0.5cm}


\newrgbcolor{lightblue}{0. 0. 0.80}
\newrgbcolor{white}{1. 1. 1.}
\newrgbcolor{whiteblue}{.80 .80 1.}


\begin{poster}

%%%%%%%%%%%%%%%%%%%%%
%%% Header
%%%%%%%%%%%%%%%%%%%%%
\begin{center}
\begin{pcolumn}{0.98}

\pbox{0.95\textwidth}{}{linewidth=2mm,framearc=0.3,linecolor=lightblue,fillstyle=gradient,gradangle=0,gradbegin=white,gradend=whiteblue,gradmidpoint=1.0,framesep=1em}{

%%% Unisiegel
\begin{minipage}[c][9cm][c]{0.1\textwidth}
  \begin{center}
    %\includegraphics[width=7cm,angle=0]{}
  \end{center}
\end{minipage}
%%% Titel
\begin{minipage}[c][17cm][c]{0.78\textwidth}
  \begin{center}
    {\sc \Huge A PSF photometry tool for NASA's Kepler, K2, and TESS missions}\\[10mm]
      {\Large \textbf{Jos\'e Vin\'icius de Miranda Cardoso}$^{1, 2, 4}$, \textbf{Geert Barentsen}$^{1, 2}$, \textbf{Ben Montet}$^{3, \star}$, \\ \textbf{Ann Marie Cody}$^{1, 2}$, \textbf{Christina Hedges}$^{1, 2}$ and \textbf{Michael Gully-Santiago}$^{1, 2}$ \\[7.5mm]
    $^{1}$NASA Ames Research Center, Moutain View, CA, USA\\
    $^{2}$Bay Area Environmental Research Institute, Petaluma, CA, USA \\
    $^{3}$Department of Astrophysics, University of Chicago, IL, USA \\
    $^{4}$Federal University of Campina Grande, Campina Grande, Brazil \\
    $^{\star}$ NASA Sagan Fellow}
  \end{center}
\end{minipage}
%%% GK-Logo
\begin{minipage}[c][9cm][c]{0.1\textwidth}
  \begin{center}
    %\includegraphics[width=7cm,angle=0]{}
  \end{center}
\end{minipage}

}
\end{pcolumn}
\end{center}


\vspace*{2cm}



%%%%%%%%%%%%%%%%%%%%%
%%% Content
%%%%%%%%%%%%%%%%%%%%%
\begin{center}
\begin{pcolumn}{0.32}
\pbox{0.9\textwidth}{65cm}{linewidth=2mm,framearc=0.1,linecolor=lightblue,fillstyle=gradient,gradangle=0,gradbegin=white,gradend=white,gradmidpoint=1.0,framesep=1em}{

\begin{center}\pbox{0.8\textwidth}{}{linewidth=2mm,framearc=0.1,linecolor=lightblue,fillstyle=gradient,gradangle=0,gradbegin=white,gradend=whiteblue,gradmidpoint=1.0,framesep=1em}{\begin{center}\textbf{Introduction}\end{center}}\end{center}
\vspace{0.75cm}

\begin{itemize}
    \item NASA's Kepler and K2 missions have been delivering high-precision time series data for a wide range of stellar types
    \item However, the existing pipelines~\citep{Luger, Vandenburg} tend to focus on studying isolated stars using simple aperture photometry, performing sub-optimally in crowded fields
    \item Crowded fields are a challenge for Kepler and K2 and it will also be for TESS
primarily because of the low resolution and spacecraft motion (in case of K2)
    \item Although Point Spread Function (PSF) photometry methods has been studying in~\citep{anderson:2000, libralato:2017}
    \item To address this issue, we present a PSF photometry toolkit for Kepler and K2
\end{itemize}

\vspace{0.75cm}
\begin{center}\pbox{0.8\textwidth}{}{linewidth=2mm,framearc=0.1,linecolor=lightblue,fillstyle=gradient,gradangle=0,gradbegin=white,gradend=whiteblue,gradmidpoint=1.0,framesep=1em}{\begin{center}\textbf{Methods}\end{center}}\end{center}
\vspace{0.75cm}

    \begin{center}
        \textbf{Fitting multiple PSFs jointly}
    \end{center}

    Consider an experiment that outputs an image with $m$ ($m$ known) stellar objects as a collection of $n$ independent \emph{non-identically} distributed random variables $\bm{Y} \triangleq \{Y_i\}_{i=1}^{n}$ (pixels), each of which has expected value $\mathbb{E}\left[Y_i\right] = \sum_{j=1}^{m}\lambda_i(\bm{\Theta}_j)$, where $\lambda_i$ is the PSF model at the $i$-th pixel, $\bm{\Theta}_j$ is a random vector
    that encondes the information about, say, flux and center position of the $j$-th star.


\begin{align}
   P\left(\bm{Y} = \bm{y} \Bigr| \left\{\bm{\Theta}_j\right\}_{j=1}^{m} = \left\{\bm{\theta}_j\right\}_{j=1}^{m}\right) = \exp\left({-\sum_{i=1}^{n}\sum_{j=1}^{m}\lambda_i(\bm{\theta}_j)}\right)\prod_{i=1}^{n}\dfrac{\left(\sum_{j=1}^{m}\lambda_i\left(\bm{\theta}_j\right)\right)^{y_i}}{y_i!}.
\end{align}

Perhaps of more practical interest is the log likelihood function
\begin{equation}
    \log  P\left(\bm{Y} = \bm{y} \Bigr| \left\{\bm{\Theta}_j\right\}_{j=1}^{m} = \left\{\bm{\theta}_j\right\}_{j=1}^{m}\right) = \sum_{i=1}^{n}\left(- \sum_{j=1}^{m}\lambda_i(\bm{\theta}_j) + y_i\log\sum_{j=1}^{m}\lambda_i(\bm{\theta}_j) - \log y_i !\right).
\end{equation}

    Then, the Maximum Likelihood Estimator (MLE) can be formulated as the following optimization problem
\begin{align}
    \bm{\theta}^{*}(\bm{y}) = \argmin_{\bm{\theta} \in \Lambda} \sum_{i=1}^{n}\left(\sum_{j=1}^{m}\lambda_i(\bm{\theta}_j) - y_i\log\sum_{j=1}^{m}\lambda_i(\bm{\theta}_j)\right),
\end{align}
which is solved numerically. Furthermore, we add prior knownledge on the stars positions, fluxes, and
sky background, as prior densities, which are often taken to be uniform on $\Lambda$.

    \begin{center}
        \textbf{Estimating uncertainties using the Cram\'er-Rao Lower Bound}
    \end{center}

Uncertainties on the fitted values are computed using the Cram\'er-Rao Lower Bound. Mathematically,

        \begin{equation}
            cov(\bm{\theta}^{*}(\bm{Y})) \leq \left(\mathbb{E}_{\bm{\theta}}\left[\nabla_{\bm{\theta}}\log p(\bm{Y} | \bm{\theta})\left[\nabla_{\bm{\theta}}\log p(\bm{Y} | \bm{\theta}) \right]^{T}  \right]\right)^{-1}\Bigr|_{\substack{\bm{\theta}=\bm{\theta}^{*}(\bm{y})}}
        \end{equation}

}
\end{pcolumn}
\begin{pcolumn}{0.32}
\pbox{0.9\textwidth}{65cm}{linewidth=2mm,framearc=0.1,linecolor=lightblue,fillstyle=gradient,gradangle=0,gradbegin=white,gradend=white,gradmidpoint=1.0,framesep=1em}{

%%% Section
\begin{center}\pbox{0.8\textwidth}{}{linewidth=2mm,framearc=0.1,linecolor=lightblue,fillstyle=gradient,gradangle=0,gradbegin=white,gradend=whiteblue,gradmidpoint=1.0,framesep=1em}{\begin{center}\textbf{Crowded K2 Clusters}\end{center}}\end{center}\vspace{1.25cm}


\begin{center}\pbox{0.8\textwidth}{}{linewidth=2mm,framearc=0.1,linecolor=lightblue,fillstyle=gradient,gradangle=0,gradbegin=white,gradend=whiteblue,gradmidpoint=1.0,framesep=1em}{\begin{center}\textbf{Kepler Faint Stars}\end{center}}\end{center}\vspace{1.25cm}
}


\end{pcolumn}
\begin{pcolumn}{0.32}
\pbox{0.9\textwidth}{65cm}{linewidth=2mm,framearc=0.1,linecolor=lightblue,fillstyle=gradient,gradangle=0,gradbegin=white,gradend=white,gradmidpoint=1.0,framesep=1em}{

\vspace{0.5cm}\begin{center}\pbox{0.8\textwidth}{}{linewidth=2mm,framearc=0.1,linecolor=lightblue,fillstyle=gradient,gradangle=0,gradbegin=white,gradend=whiteblue,gradmidpoint=1.0,framesep=1em}{\begin{center}\textbf{Analysis}\end{center}}\end{center}\vspace{0.5cm}

\begin{center}\pbox{0.8\textwidth}{}{linewidth=2mm,framearc=0.1,linecolor=lightblue,fillstyle=gradient,gradangle=0,gradbegin=white,gradend=whiteblue,gradmidpoint=1.0,framesep=1em}{\begin{center}\textbf{Conclusions}\end{center}}\end{center}

%%% References
\bibliographystyle{unsrt}
\bibliography{poster.bib}


}
\end{pcolumn}
\end{center}

\end{poster}

\end{document}

